\documentclass{adonis}
\usepackage{hyperref}
\usepackage{lipsum}
\usepackage{tabularx}
\usepackage{xurl}

\title{Adonis: a guide}
\author{Nicholas Mamo \textsuperscript{1}}
\affiliation{\textsuperscript{1} Independent}
\correspondence{nicholasmamo@gmail.com}
\date{\today}
\runningauthor{Nicholas Mamo}
\runningtitle{Adonis: a guide}

\abstract{
	Academic writing does not have to be drab, and neither does the academic writing template.
	Not in academia, however.
	In academia, form normally follows function.
	The { \upshape Adonis } template is my attempt to rectify that shortcoming.
	I designed it as a no-frills but elegant template, built on the basic article template.
	At its core, three principles: simplicity, readability and aesthetic.
	This guide also plays three roles: to serve as an illustrative example, to guide you in using this template, and to document my changes.
}

\begin{document}
	\maketitle
	
	\section{Introduction}
	
		The best way to appreciate a good template is to force yourself to write with a bad template.
		This is the \textit{Adonis} template.
		Its design stems from a personal experience.
		I was writing a manuscript that I had been planning for months, about a project that I had been developing for years, and on a subject I adored.
		And I dreaded every minute I spent drafting the manuscript.
		It took a while until I realized why: the template felt entirely off-putting.
		
		The template should elevate the writing, not diminish it.
		In academia, however, form normally follows function.
		Sometimes it feels like publishers deliberately diminish the form in a vain attempt to elevate the function: the words, the science, and nothing else.
		Having a good writing environment matters.
		To design the \textit{Adonis} template, I followed three principles:
		
		\begin{itemize}
			\item Simplicity, by which I mean several things.
				  I mean that I wanted the template to be simple for me to develop, lest it turn into an exercise in procrastination.
				  I also mean that it should be easy for you, the writer, to use and adapt.
				  Above all, I mean that it should be easy for your reader to consume.
			
			\item Readability, by which I mean the entire writing process.
				  The template had to make it easy to draft manuscripts, revise and read.
			
			\item Aesthetic, by which I mean elegant.
				  Simple and readable \LaTeX templates abound, but when I looked, I found most to favour function over form.
		\end{itemize}
	
		I hope that you find this template to elevate both form and function, and that it proves it possible for the two to co-exist.
		The rest of this guide documents design considerations for the layout, typography and other elements.
	
	\section{Layout}
	
		The \textit{Adonis} template is based on the base \textit{article} template.
		The paper dimensions are those of an A4 paper.
		Many changes concern its layout, and most add white space.
		The margins are wider, notably on the sides, but also at the top and bottom.
		The extra space serves a dual purpose: an obvious aesthetic one, and a more functional one.
		The added space makes it easier to add notes in the margin.
		
		Heading and footer: Not on front page (elegance)
		
		\subsection{Options}
		
			\subsubsection{Two columns}
			
				\begin{verbatim}
					\documentclass[twocolumn]{elegantarticle}
				\end{verbatim}
			
			\subsubsection{Wide}
			
				\begin{verbatim}
					\documentclass[wide]{elegantarticle}
				\end{verbatim}
	
		\subsection{Frontmatter}
		
			Optional
	
	\section{Typography}
	
		\subsection{Headings}
		
			\subsubsection{Section}
			
			\subsubsection{Subsection}
			
			\subsubsection{Subsubsection}
			
	\section{Other elements}
	
		\begin{table*}[t!]
			\begin{tabularx}{\linewidth}{ l l X }
				\textbf{Version} & \textbf{Date} & \textbf{Changelog} \\ \hline
				0.1 & April 15, 2023 & Initial release \\
			\end{tabularx}
			\caption{The template's version history.}
			\label{table: version history}
		\end{table*}
			
	\section{Conclusion}
	
	
		Repository (including citation) \cite{repository}
		
	\begin{thebibliography}{9}
		\bibitem{repository}
		Adonis template. Nicholas Mamo (2023). \url{https://github.com/NicholasMamo/adonis-template}
	\end{thebibliography}
	
\end{document}